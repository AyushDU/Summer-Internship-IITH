\documentclass[journal,12pt,twocolumn]{IEEEtran}

\usepackage{setspace}
\usepackage{gensymb}

\singlespacing


\usepackage[cmex10]{amsmath}

\usepackage{amsthm}

\usepackage{mathrsfs}
\usepackage{txfonts}
\usepackage{stfloats}
\usepackage{bm}
\usepackage{cite}
\usepackage{cases}
\usepackage{subfig}

\usepackage{longtable}
\usepackage{multirow}

\usepackage{enumitem}
\usepackage{mathtools}
\usepackage{steinmetz}
\usepackage{tikz}
\usepackage{circuitikz}
\usepackage{verbatim}
\usepackage{tfrupee}
\usepackage[breaklinks=true]{hyperref}
\usepackage{graphicx}
\usepackage{tkz-euclide}

\usetikzlibrary{calc,math}
\usepackage{listings}
    \usepackage{color}                                            %%
    \usepackage{array}                                            %%
    \usepackage{longtable}                                        %%
    \usepackage{calc}                                             %%
    \usepackage{multirow}                                         %%
    \usepackage{hhline}                                           %%
    \usepackage{ifthen}                                           %%
    \usepackage{lscape}     
\usepackage{multicol}
\usepackage{chngcntr}

\DeclareMathOperator*{\Res}{Res}

\renewcommand\thesection{\arabic{section}}
\renewcommand\thesubsection{\thesection.\arabic{subsection}}
\renewcommand\thesubsubsection{\thesubsection.\arabic{subsubsection}}

\renewcommand\thesectiondis{\arabic{section}}
\renewcommand\thesubsectiondis{\thesectiondis.\arabic{subsection}}
\renewcommand\thesubsubsectiondis{\thesubsectiondis.\arabic{subsubsection}}


\hyphenation{op-tical net-works semi-conduc-tor}
\def\inputGnumericTable{}                                 %%

\lstset{
%language=C,
frame=single, 
breaklines=true,
columns=fullflexible
}
\begin{document}


\newtheorem{theorem}{Theorem}[section]
\newtheorem{problem}{Problem}
\newtheorem{proposition}{Proposition}[section]
\newtheorem{lemma}{Lemma}[section]
\newtheorem{corollary}[theorem]{Corollary}
\newtheorem{example}{Example}[section]
\newtheorem{definition}[problem]{Definition}

\newcommand{\BEQA}{\begin{eqnarray}}
\newcommand{\EEQA}{\end{eqnarray}}
\newcommand{\define}{\stackrel{\triangle}{=}}
\bibliographystyle{IEEEtran}
\providecommand{\mbf}{\mathbf}
\providecommand{\pr}[1]{\ensuremath{\Pr\left(#1\right)}}
\providecommand{\qfunc}[1]{\ensuremath{Q\left(#1\right)}}
\providecommand{\sbrak}[1]{\ensuremath{{}\left[#1\right]}}
\providecommand{\lsbrak}[1]{\ensuremath{{}\left[#1\right.}}
\providecommand{\rsbrak}[1]{\ensuremath{{}\left.#1\right]}}
\providecommand{\brak}[1]{\ensuremath{\left(#1\right)}}
\providecommand{\lbrak}[1]{\ensuremath{\left(#1\right.}}
\providecommand{\rbrak}[1]{\ensuremath{\left.#1\right)}}
\providecommand{\cbrak}[1]{\ensuremath{\left\{#1\right\}}}
\providecommand{\lcbrak}[1]{\ensuremath{\left\{#1\right.}}
\providecommand{\rcbrak}[1]{\ensuremath{\left.#1\right\}}}
\theoremstyle{remark}
\newtheorem{rem}{Remark}
\newcommand{\sgn}{\mathop{\mathrm{sgn}}}
\providecommand{\abs}[1]{\vert#1\vert}
\providecommand{\res}[1]{\Res\displaylimits_{#1}} 
\providecommand{\norm}[1]{\Vert#1\rVert}
%\providecommand{\norm}[1]{\lVert#1\rVert}
\providecommand{\mtx}[1]{\mathbf{#1}}
\providecommand{\mean}[1]{E[ #1 ]}
\providecommand{\fourier}{\overset{\mathcal{F}}{ \rightleftharpoons}}
%\providecommand{\hilbert}{\overset{\mathcal{H}}{ \rightleftharpoons}}
\providecommand{\system}{\overset{\mathcal{H}}{ \longleftrightarrow}}
	%\newcommand{\solution}[2]{\textbf{Solution:}{#1}}
\newcommand{\solution}{\noindent \textbf{Solution: }}
\newcommand{\cosec}{\,\text{cosec}\,}
\providecommand{\dec}[2]{\ensuremath{\overset{#1}{\underset{#2}{\gtrless}}}}
\newcommand{\myvec}[1]{\ensuremath{\begin{pmatrix}#1\end{pmatrix}}}
\newcommand{\mydet}[1]{\ensuremath{\begin{vmatrix}#1\end{vmatrix}}}
\numberwithin{equation}{subsection}
\makeatletter
\@addtoreset{figure}{problem}
\makeatother
\let\StandardTheFigure\thefigure
\let\vec\mathbf
\renewcommand{\thefigure}{\theproblem}
\def\putbox#1#2#3{\makebox[0in][l]{\makebox[#1][l]{}\raisebox{\baselineskip}[0in][0in]{\raisebox{#2}[0in][0in]{#3}}}}
     \def\rightbox#1{\makebox[0in][r]{#1}}
     \def\centbox#1{\makebox[0in]{#1}}
     \def\topbox#1{\raisebox{-\baselineskip}[0in][0in]{#1}}
     \def\midbox#1{\raisebox{-0.5\baselineskip}[0in][0in]{#1}}
\vspace{3cm}
\title{ASSIGNMENT-2}
\author{Ojaswa Pandey}
\maketitle
\newpage
\bigskip
\renewcommand{\thefigure}{\theenumi}
\renewcommand{\thetable}{\theenumi}
Download all python codes from 
\begin{lstlisting}
https://github.com/behappy0604/Assignment2
\end{lstlisting}
%
and latex-tikz codes from 
%
\begin{lstlisting}
https://github.com/behappy0604/Assignment2
\end{lstlisting}
%
\section{Question No. 2.39}
Construct a quadrilateral MORE where $MO = 6, OR = 4.5, \angle M = 60 \degree, \angle O = 105 \degree$ and $\angle R = 105 \degree$.
%
\section{SOLUTION}
For this quadrilateral MORE we have,
\begin{align}
\angle M +\angle O +\angle R = 60\degree + 105\degree + 105\degree=270\degree,
\end{align}

\begin{enumerate}
    \item Now on calculating, we get
\begin{align}
&\implies \angle E + 270\degree  = 360\degree,
\\
&\implies \angle E = 90\degree\label{eq1}
\end{align}
 \item Now taking sum of all the angles given and \eqref{eq1}  \text{we get}
\begin{align}
\angle M + \angle O +\angle R +\angle E =360\degree
\end{align}
So construction of given quadrilateral is possible as sum of all the angles is equal to $360\degree$.
\\
 \item Now,Using cosine formula in $\triangle MOR$ we can find RM:
\begin{multline}
\implies {\norm{\vec{R}-\vec{M}}^2}=
\\
{\norm{\vec{M}-\vec{O}}^2}+ \norm{\vec{O}-\vec{R}}^2-2\times\norm{\vec{M}-\vec{O}} \times \norm{\vec{O}-\vec{R}}\cos{O}
\end{multline}
\begin{align}
&\implies RM=8.38
\end{align}
\item Also in $\triangle MOR$, Let \angle OMR =$\theta$, \angle MOR =$\beta$, \angle ORM = $\gamma$.Now using sine formula in $\triangle MOR$ we have
\begin{align}
\frac{{\sin \theta}}{OR} = \frac{{\sin \beta}}{RM} = \frac{{\sin \gamma}}{MO}
\end{align}
\begin{align}
\theta=\sin^{-1} (0.5186);
\\
\theta=\angle OMR= 31.24 \degree ;
\end{align}
\item Now polar coordinates of vertex R of $\triangle MOR$ be
\\$(RM \cos \theta,RM\sin \theta)$, we get
\begin{align}
R(8.38\times \cos31.24 ,8.38\times \sin31.24)
\\
\implies \vec{R}= \myvec{7.16\\4.35}
\end{align}
\item Now in $\triangle MER$,we get
\begin{align}
\angle EMR= 28.76\degree
\end{align}
\item Considering the polar coordinates of E of $\triangle MER$ and solving we get,
\begin{align}
\implies \vec{E}= \myvec{3.67 \\ 6.36}
\end{align}
    \item Now, we have the coordinate of vertices M,O,R,E as
\vec{M}= \myvec{0 \\ 0}, \vec{O}= \myvec{6 \\ 0},  \vec{R}= \myvec{7.16 \\ 4.35}, \vec{E}= \myvec{3.67 \\ 6.36}.
    \item On constructing the given quadilateral on python we get:
\end{enumerate}
\numberwithin{figure}{section}
\begin{figure}[!ht]
\centering
\includegraphics[ width=\columnwidth, height=5 cm]{quadilateral more.png}
\caption{Quadrilateral MORE}
\label{fig:Quadrilateral MIST}	
\end{figure}
\end{document}
